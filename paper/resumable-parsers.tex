% -*- coding: utf-8; -*-
\documentclass[english,submission]{programming}
\usepackage{listings}
\usepackage[backend=biber]{biblatex}
\addbibresource{example.bib}
\begin{document}
  \lstset{language=Scala, basicstyle=\small}
  \title{Resumable Parser Combinators}
  \author{Noel Welsh}
  \authorinfo{is the author of this paper. Contact him at
    \email{noel@noelwelsh.com}.}
  \affiliation[a]{Inner Product LLC}
  \keywords{programming journal, paper formatting, submission preparation} % please provide 1--5 keywords
  \paperdetails{
    %% perspective options are: art, sciencetheoretical, scienceempirical, engineering.
    %% Choose exactly the one that best describes this work. (see 2.1)
    perspective=art,
    %% State one or more areas, separated by a comma. (see 2.2)
    %% Please see list of areas in http://programming-journal.org/cfp/
    %% The list is open-ended, so use other areas if yours is/are not listed.
    area={Parsing},
    %% You may choose the license for your paper (see 3.)
    %% License options include: cc-by (default), cc-by-nc
    license=cc-by,
  }
  % \begin{CCSXML}
  % <ccs2012>
  % <concept>
  % <concept_id>10002944.10011122.10003459</concept_id>
  % <concept_desc>General and reference~Computing standards, RFCs and guidelines</concept_desc>
  % <concept_significance>300</concept_significance>
  % </concept>
  % <concept>
  % <concept_id>10010405.10010476.10010477</concept_id>
  % <concept_desc>Applied computing~Publishing</concept_desc>
  % <concept_significance>300</concept_significance>
  % </concept>
  % </ccs2012>
  % \end{CCSXML}

  % \ccsdesc[300]{General and reference~Computing standards, RFCs and guidelines}
  % \ccsdesc[500]{Applied computing~Publishing}

  \maketitle

  % Please always include the abstract.
  % The abstract MUST be written according to the directives stated in
  % http://programming-journal.org/submission/
  % Failure to adhere to the abstract directives may result in the paper
  % being returned to the authors.
  \begin{abstract}
  \end{abstract}


  \section{Introduction}

Many languages allow string interpolation: a string literal can contain placeholders that indicate where the value of an expression should be substituted into the string. For example, in Scala we can write

\begin{lstlisting}
  val name = "Noel"

  val hi = s"Hello $name!"
\end{lstlisting}

and \lstinline{hi} will have the value \lstinline{"Hello Noel!"}.

Some languages, including Scala, allow extensible string interpolation. In Scala the character in front of the interpolated string, \texttt{s} in the example above, determines how the string is processed. The details, which are not important here, are given in the [documentation for the `StringContext` API][StringContext]. Additionally, the result of a string interpolation does not necessarily have to be a string. Interpolation can evaluate to any type. This means that string interpolation can be used to embed domain specific languages (DSLs) within Scala, with interpolation functioning as the interface between the DSL and the Scala host language. Lisp programmers will recognize string interpolation as a form of quasi-quote and unquote.

This is fine in theory, but there is a problem: how do we parse our embedded DSL when the parsing may be interrupted at any time with an interpolated value? This would be straightforward if only strings could be supplied as interpolated values. In this setting we could simply render everything to a string and then parse the result. However a major advantage of creating an embedded DSL is that we can pass structured data from the host language into the embedded language, and therefore working only with strings is not sufficient.

In this paper we present \emph{resumable parser combinators}, an extension of parser combinators that allows parsing to be interrupted, parsed values to be injected, and parsing resumed with additional input. We describe the design and implementation of the library. etc.

Four main components:

\begin{enumerate}
  \item a model for resumable parser combinators allowing suspending parsing, injecting parsed values, and resuming parsing;
  \item methods to define semantics for suspending and resuming parser combinators;
  \item implementation technique; and
  \item reusable general principles through which we can view the design and implementation.
\end{enumerate}

  \section{Parser Combinators}

Parser combinators are a well established technique for creating a parser in a composable in a composable . Their ease of use and simplicity of implementation make them ideal for ...

We start with a description of a basic parser combinator library, and then describe how we extend the basic library to support resumable parsing.


\subsection{Parser}

The \lstinline{Parser} type implements a basic parser combinator library.



  \section{Design Constraints}

  \section{Design}

  The overall design is summarized by Listing~\ref{lst:parser}.


  \begin{lstlisting}[frame=lines, caption={The \texttt{Parser} type}, float=*, label=lst:parser]
trait Parser[A]:
  def parse(input: String): Parser.Result[A]

  def resume(using
    semigroup: Semigroup[A],
    ev: String =:= A
  ): Suspendable[A, A]

  def resumeWith(f: String => A)(using
      semigroup: Semigroup[A]
  ): Suspendable[A, A]

  def commit[S]: Suspendable[S, A]

object Parser:
  enum Result[A]:
    case Epsilon(input: String, start: Int)
    case Committed(input: String, start: Int, offset: Int)
    case Continue(result: A, input: String, start: Int)
    case Success(result: A, input: String, start: Int, offset: Int)
  \end{lstlisting}
  \printbibliography
\end{document}
